\documentclass[a4paper,notitlepage,dvipsnames]{scrreprt}
\usepackage{listings} % For source code presentation
\usepackage{needspace}
\usepackage{color}
\usepackage[hyphens]{url}
\usepackage{pifont}
\usepackage{multicol}
\usepackage{blindtext}
\usepackage{braket}
\usepackage{amsmath}
\usepackage[framemethod=tikz]{mdframed}
\usepackage[top=3cm, bottom=3cm]{geometry}
\usepackage{hyperref}

% Set up clickable links in contents
\hypersetup{
    colorlinks=true,
    linktoc=all,
    linkcolor=blue,
    urlcolor=blue
}
%	citecolor=black,
%	filecolor=black,
%	linkcolor=black,
%	urlcolor=black
%}

% Use paragraph spacing, rather than indentation. This is generally easier
% to read on a screen (especially for technical documentation).
\usepackage{parskip}
% define custom colors used in the description of keywords
\usepackage{xcolor}
\definecolor{mred}{rgb}{0.9, 0.1, 0.1}
\definecolor{oblue}{rgb}{0.1, 0.1, 0.9}

% An itemize with a bit less space floating around
\newenvironment{packed_enum}{
	\begin{enumerate}
		\setlength{\itemsep}{1pt}
		\setlength{\parskip}{0pt}
		\setlength{\parsep}{0pt}
	}{\end{enumerate}}
\newenvironment{packed_itemize}{
	\begin{itemize}
		\setlength{\itemsep}{1pt}
		\setlength{\parskip}{0pt}
		\setlength{\parsep}{0pt}
	}{\end{itemize}}

% Default formatting for code
\definecolor{commentgreen}{rgb}{0,0.6,0}
\lstset{
	keywordstyle=\color{blue},
	commentstyle=\color{commentgreen},
	language=Fortran,
	tabsize=4,
	basicstyle=\footnotesize\ttfamily,
	morekeywords={HElement_t,pure,elemental,supermodule,ubound,lbound,abstract,protected},
	belowskip=-3pt,
	lineskip=-1pt    % Bring the lines a bit closer together (eliminate spaces)
}

% Define INI-file syntax
\lstdefinelanguage{ini}
{
    morecomment=[s][\color{blue}]{[}{]},
    morecomment=[l]{\#},
	morekeywords={integer,real},
}

% Headitem is a new command for the description environment, which is roughly
% equivalent to "\item[title] \hfill \\ ...", but ensuring that space is used
% correctly. May need to tweak needspace to be better.
\newcommand\headitem[1]{\needspace{1.5\baselineskip}\item[{\boldmath #1 \nopagebreak}] \hfill \\ \nopagebreak}
\newcommand\headit[1]{\needspace{1.5\baselineskip}\item[\textnormal{\itshape #1 \nopagebreak}] \hfill \\ \nopagebreak}
\newcommand\codeitem[1]{\needspace{1.5\baselineskip}\item[\textnormal{\ttfamily #1 \nopagebreak}] \hfill \\ \nopagebreak}

% Use a defined command instead of lstinline, as it breaks texmaker
% syntax highlighting
\let\code\lstinline

% A nice box for warnings and exclamations
\newenvironment{warningbox}
	{\par\begin{mdframed}[%
		linewidth = 2pt, %
	    linecolor = red, %
	    roundcorner = 6pt, %
		backgroundcolor = gray!20
	]\begin{list}{}{\leftmargin=1cm
			           \labelwidth=\leftmargin}\item[\Large\ding{43}]}
	{\end{list}\end{mdframed}\par}

% A nice mathematics abs command
\providecommand{\abs}[1]{\ensuremath{\left\lvert#1\right\lvert}}

% Modify TOC command so it can start on the same page as the title
\makeatletter
\newcommand*{\toccontents}{\@starttoc{toc}}
\makeatother
\setcounter{tocdepth}{3}

\begin{document}

\author{Simon Smart, Nick Blunt, Kai Guther, Oskar Weser, Werner Dobrautz and George Booth}
\title{Unravelling the mysteries of NECI}
\subtitle{The $n$-Electron Configuration Interaction solver}
\maketitle

\toccontents

\chapter{Using NECI}
\section{Getting into the game}
\subsection{Getting the code}
	The NECI repository is stored on bitbucket. To gain access you need to be
	invited. Contact one of the repository administrators
	[Simon Smart (\url{simondsmart@gmail.com}), George Booth
	(\url{george.booth24@gmail.com}) and Nick Blunt (\url{nsb37@cam.ac.uk})] who will
	invite you. If you already have a bitbucket account let the repository
	administrators know the email address associated with your account.

	You will receive an invitation email. Please accept this invitation, and
	create a bitbucket account as prompted if necessary.

	To gain access to the NECI repository, an ssh key is required. This can
	be generated on any linux machine using the command\footnote{%
		\code{ssh-keygen} can also generate DSA keys. Some ssh clients and
		servers will reject DSA keys longer than 1024 bits, and 1024 bits is
		currently on the margin of being crackable. As such 2048 bit RSA keys
		are preferred. Top secret this code is. Probably. Apart from the master branch which hosted for all on github. And in molpro.
		And anyone that wants it obviously.
	}
	\begin{lstlisting}[gobble=4]
		ssh-keygen -t rsa -b 2048
	\end{lstlisting}%
	This will create a private (\code{~/.ssh/id_rsa}) and a public key
	file (\code{~/.ssh/id_rsa.pub}).

	The private key must be kept private. On the bitbucket homepage, go to
	account settings (accessible from the top-right	of the main page), and
	navigate to ``SSH keys''. Click ``Add key'' and add the contents of the
	public key. This will give you access to the repository.

	You can now clone the code into a new directory using the command
	\begin{lstlisting}[gobble=4]
		git clone git@bitbucket.org:neci_developers/neci.git [target_dir]
	\end{lstlisting}

\subsection{Required libraries}
	NECI requires some external software and library support to operate:
	\begin{description}
		\headitem{MPI}
			For builds of NECI intended to be run in parallel, an
			implementation of MPI is required. NECI has been heavily tested
			with OpenMPI, and MPICH2 and its derivatives (IBM MPI, Cray MPI,
			and Intel MPI).
		\headitem{Linear algebra}
			NECI makes use of the linear algebra routines normally contained in
			BLAS/LAPACK. There are a number of different packages which provide
			these routines, and are optimised for different compilers and
			platforms. NECI has been built and tested with either the AMD Core
			Math Library (ACML), the Intel Math Kernel Library (MKL), or the
			more general Basic Linear Algebra Subprograms (BLAS)/Linear Algebra
			Package (LAPACK) combination.
		\headitem{HDF5 (optional)}
			To make use of the structured HDF5 format for reading/writing
			POPSFILES (files storing the population of walkers, and other
			information, to restart calculations). This library should be
			built with MPI and fortran support
			(\code{--enable-parallel} \code{--enable-fortran}\
			\code{--enable-fortran2003}).
		\headitem{FFTW (optional)}
			A small number of options in NECI, which are not enabled by
			default, require Fast Fourier Transforms. If these are re-enabled
			then the Fastest Fourier Transform in the West (FFTW3) library
			is required.
	\end{description}
	If combinations of these choices are made other than those most commonly
	used then either the configuration files, or the resultant Makefile, will
	need to be modified.

	On the majority of machines available to the Alavi group and department,
	the compilation environment is managed using the
	\code[language=bash]{module} command. Documentation for that command
	is available from the command \code[language=bash]{module help}. The
	most commonly used command is to load a module, using the command
	\begin{lstlisting}[gobble=4,language=bash]
		module load <module_name>
	\end{lstlisting}
	Installing and configuring the module system on private machines is far
	beyond the scope of this document. Configuring your user account to use
	modules may required modifications to your \code{.bashrc} file,
	depending on the local machine configuration. Please contact the local
	IT administrators or Simon Smart for further advice.

	A number of standard combinations of modules present themselves. Where an
	asterisk is presented, any version of the module can be used. Where the
	version is under specified, the latest module should be used. This will
	occur by default.
	\begin{description}
		\headitem{gfortran (Cambridge)}
			\code{mpi/mpich2/gnu/1.4.1p1 acml/64/gfortran/*/up}
		\headitem{ifort (Cambridge)}
			\code{ifort/64 mpi/openmpi/64/intel12 acml/64/ifort/*/up}
		\headitem{PGI (Cambridge)}
			\code{pgi/64 mpi/openmpi/64/pgi12 lapack/64/pgi} \linebreak
			The lapack module might not be available to all users. Please
			contact Simon Smart if required.
		\headitem{ifort (Max Planck FKF)}
			\code{ifort mpi.intel}\linebreak
			Note that the MKL library is included in the ifort module.
		\headitem{ifort (hydra)}
			\code{git intel mkl mpi.ibm hdf5-mpi cmake}
	\end{description}

\subsection{Building the code (using CMake)}
	There are two ways of building NECI. The recommended approach is to use
	the cmake build system. For legacy purposes, and for more explicitly
	customised build configurations, the older Makefile system may also
	be used.

	CMake allows building the code in a separate directory. One directory
	should be used per configuration that is to be built. This module can
	be a subdirectory of the NECI directory, or otherwise.
	With the \code{build} directory as the current working directory,
	execute
	\begin{lstlisting}[gobble=4]
	cmake [-DCMAKE_BUILD_TYPE=<type>] <path_to_neci>
	\end{lstlisting}
	pointing CMake at the root directory of the cloned NECI repository.

	At this point CMake will automatically configure NECI according to the
	currently loaded modules, and available compilers and libraries. If a
	different set of modules or compilers are to be used a fresh directory
	should be initialised (or equivalently all contents of the directory
	deleted).

	By default CMake will configure a Release build. If this is not desired
	an alternative build type may be specified by setting the
	\code{CMAKE_BUILD_TYPE} above. The available options are \code{Debug} (no
	optimisations, all checking enabled), \code{Release} (optimisations
	enabled), \code{RelWithDebInfo} (same as release, but with debugging
	symbols retained) or \code{Cluster} (inter-procedural optimisations
	enabled, with very long compile times).

	\begin{mdframed}[ %
		linewidth = 2pt, %
		linecolor = red, %
		roundcorner = 6pt, %
		leftmargin = 10, %
		rightmargin = 10, %
		backgroundcolor = gray!20
	]
	If there is an available HDF5 library, which is compiled with support for
	MPI and for the Fortran compiler in use, then CMake will happily make use
	of it. Otherwise support for HDF5 POPSFILES will be disabled by default.

	The CMake configuration for NECI contains the functionality to download,
	compile and use HDF5. To do this run CMake with the
    \code{-DENABLE_BUILD_HDF5=ON}.
	\end{mdframed}

	The code is then built using the command
	\begin{lstlisting}[gobble=4]
		make [-j [n]] [neci|kneci|dneci|mneci]
	\end{lstlisting}
	The optional flag \code{-j} specifies that the build should be
	performed in parallel (with up to an optional number, \code{n},
	threads).

	The final argument specifies whether the normal code (\code{neci})
	should be built, the complex code (\code{kneci}) the double run
	code (\code{dneci}) or the multi run code (\code{mneci}) are
	built. If not specified, all targets are built.

\subsubsection{Options}
    Whilst every effort has been made to provide NECI with sensible default options, the user
    may wish to play around further. To (dis)able an option, the following should be passed as
    an argument to cmake:
    \begin{lstlisting}[gobble=4]
        -DENABLE_<option>=<(ON|OFF)>
    \end{lstlisting}
    The following options are available. Where an option is default "on", if the required libraries
    are not available, the option will be disabled and this will be noted in the build summary.
    \begin{description}
        \codeitem{BUILD\_HDF5}
            Build the hdf5 library from source, and use that instead of one provided by the system.
        \codeitem{HDF5}
            Make use of hdf5 for popsfiles (default=on).
        \codeitem{FFTW}
            Functionality requiring FFTW (default=on).
        \codeitem{MOLCAS}
            Build with the \_MOLCAS\_ flag (default=off).
        \codeitem{MPI}
            Build with parallel functionality (default=off).
        \codeitem{SHARED\_MEMORY}
            Use shared memory for storing integrals (default=on).
        \codeitem{WARNINGS}
            Compile with verbose compiler warnings (default=off).
    \end{description}

\subsection{Overriding configuration options}

    One of the aims of the CMake tool is to make build configuration as black-box as possible.
    The build system should normally detect the compilers in use automatically. The detected compilers
    may not, however, be the ones desired, or the build system may fail to find functionality that
    exists. The complier to use can be overridden by arguments passed to CMake:
    \begin{lstlisting}[gobble=4]
        cmake -DCMAKE_<lang>_COMPILER=XXX <neci_dir>
    \end{lstlisting}
    where \code{<lang>} may be \code{Fortran}, \code{CXX} or \code{C} as appropriate. This should
    specify the command to use which may be a compiler available in the environmental \code{PATH},
    or and absolute path to the compiler to use.

    Once CMake has determined the compiler to use, it determines the compilation and linker flags
    automatically. A number of overrides have been definied for NECI. These may be found in the
    \code{cmake/compiler_flags} director, and are segregated by files named according to both the
    compiler vendor and the language.

    Compiler flags are added in a specific order. The flags defined in the \code{NECI_<lang>_FLAGS}
    variable are applied to all builds, whereas those in \code{NECI_<lang>_FLAGS_<type>} are only
    applied to builds with the appropriate build type (with the exception of when type is equal to
    \code{CLUSTER} when the flags are appended to those used in \code{RELEASE} mode to enable
    extra inter-file optimisations).

    There are also flags to control how things are linked, what options are passed to the compiler
    to enable compiler warnings, and flags that depend on 32 or 64 bit builds. These should be
    self explanatory from reading the files in \code{cmake/compiler_flags}.

    If these defaults are insufficient, the compilation flags may also be overridden. Any arguments
    passed to cmake of the form
    \begin{lstlisting}[gobble=4]
        cmake -DFORCE_<lang>_FLAGS[_<type>]
    \end{lstlisting}
    override the corresponding \code{NECI_<lang>_FLAGS[_<type>]} flags. Essentially any
    \code{NECI_*} flag may be overridden on the command line with a \code{FORCE_*} flag (although
    it is possible that some of these have been accidentally ommitted from the implmentation).

\subsubsection{Toolchain files}

    Overriding all of the CMake variables on the command line is cumbersome and error prone.
    Various sets of overrides can be combined into a toolchain file, which can be passed
    to CMake:
    \begin{lstlisting}[gobble=4]
        cmake -DCMAKE_TOOLCHAIN_FILE=<toolchain_file> <neci_dir>
    \end{lstlisting}
    These toolchain files can specify the entire chain of compilers, flags and libraries if
    desired. For examples see the toolchains/ directory in the NECI repository.

    Additionally to the flags described above, the \code{CMakeForceCompiler} functionality may
    be used (over and above just setting the \code{CMAKE_<lang>_COMPILER} variable).
    These macros entirely disable the autodetection of compiler properties within
    CMake (per language), and will require all flags that are not in the
    \code{cmake/compiler_flags} directories to be specified manually. As an example:
    \begin{lstlisting}[gobble=4]
        include(CMakeForceCompiler)

        CMAKE_FORCE_C_COMPILER       ( gcc GNU )
        CMAKE_FORCE_CXX_COMPILER     ( g++ GNU )
        CMAKE_FORCE_Fortran_COMPILER ( mpif90 GNU )
    \end{lstlisting}
    This will force the use of the commands \code{gcc}, \code{g++}, and \code{mpif90}, and will
    set the \code[breaklines=true]{CMAKE_<lang>_COMPILER_ID} variable to \code{GNU} such that
    the compiler flags set in the \code[breaklines=true]{cmake/compiler_flags/GNU_*.cmake}
    files are used. Because this turns off
    any autodetection, CMake will not autodetect that the c++ standard library needs to be
    linked in to combine c++ and Fortran files. This will need to be corrected manually, using:
    \begin{lstlisting}[gobble=4]
        set( NECI_Fortran_STATIC_LINK_LIBRARIES stdc++ )
    \end{lstlisting}
    Further internally required libraries may be required. In a normal build, these are
    output in the build summary under "Implicit C++ linker flags".
    A comprehensive documented example is found in \code{toolchains/gfortran-openmpi.cmake}.

    \textbf{Archer}

    As an example, we can consider the Archer supercomputer, located at EPCC in Edinburgh. This
    machine uses compiler wrapper scripts (written by Cray) to provide much of the functionality
    automatically, but this defeats the CMake auto-configuration system. To build on archer the
    cmake command
    \begin{lstlisting}[gobble=4]
        cmake -DCMAKE_TOOLCHAIN_FILE=<neci_dir>/toolchains/archer.cmake <neci_dir>
    \end{lstlisting}

    (As we already know about Archer, we autodetect that you are running on it, and CMake will
    fail with a message containing these instructions).
	used, then the build will fail.

\subsubsection{Compilation on ADA}
	Begin by clearing out the build environment. At the terminal execute:
	\begin{lstlisting}[gobble=4]
		module purge
		module load environments/addons/cmake-2.8.2
	\end{lstlisting}
	Next, for GNU:
	\begin{lstlisting}[gobble=4]
		module load environments/programming/gcc-4.8.2
	\end{lstlisting}
	Or Intel:
	\begin{lstlisting}[gobble=4]
		module load compilers/intel/15.0.0.090
		module load mpi/openmpi/1.8.2/intel15.0-threads
	\end{lstlisting}
	then run Cmake as normal.

\subsubsection{Overriding packages required for options}

    There are a number of packages that are required to build NECI (such as something providing
    a LAPACK-like interface), or which are required to enable certain options (librt is required
    to enable shared memory).

    If the package searching fails, there are a number of variables that can be set on the CMake
    command line, or in a toolchain file as appropriate:

    \begin{description}
        \codeitem{NECI\_FIND\_<package>}
            If this is set to OFF, the package searcher will not be executed, and the associated
            option will be enabled without disabling the option.
        \codeitem{<package>\_FOUND}
            For many of the package searchers, this disables the searching from inside the package
            rather than outside. In general, the first option is preferred.
        \codeitem{<package>\_LIBRARIES}
            The libraries to be linked in for use of the specifide package. If package searching is
            disabled, then to use the package this needs to be filled in explicitly.
        \codeitem{<package>\_DEFINITIONS}
            Any additional compiler flags that are required to use the package.
        \codeitem{<package>\_INCLUDE\_PATH}
            The location of any C or C++ header files, or fortran module files to be used during
            compilation
    \end{description}

    NECI has a some special package finders, called \code{MPI_NECI}, \code{LAPACK_NECI} and
    \code{HDF5_NECI}. To a large
    extent they are just wrappers around the underlying finders provided with CMake, but they
    implement some additional logic, such as automatically substituting MKL for LAPACK, checking
    the MPI compiler being used, and providing the capacity to build hdf5 in the source tree.

    As a result, when overriding these packages, \code{MPI_NECI}, \code{LAPACK_NECI} and
    \code{HDF5_NECI} should be substituted for \code{<package>} above.

\subsection{Configuring builds (Makefile system)}
	A specific configuration for building NECI is initialised by using the
	command
	\begin{lstlisting}[gobble=4]
		./tools/mkconfig.py config_name [-g]
	\end{lstlisting}
	The configuration names correspond to the configuration files contained in
	the config directory. If the flag \code{-g} is used, then a debug
	configuration will be created, and otherwise an optimised one.

	There are a number of different configurations for
	differing systems, library and compiler setups. The following are some of
	the more important, and most likely to be used.
	\begin{description}
		\codeitem{gfortran\_simple, ifort\_simple}
			These are the basic configurations, set up for the gfortran and
			ifort compilers. For development these are the most likely
			configurations to be used. On personal machines the easiest
			environment to install is gfortran and openMPI, using the
			\code{gfortran\_simple} - see note regarding libraries below.
		\codeitem{fkf\_ifort}
			The MPI and ifort installation at the FKF in Stuttgart is different
			to that available in most locations. Use this config file to
			compile there.
		\codeitem{PC-ifort64-MPI-TARDIS, PC-ifort64-MPI-HYDRA}
			The ifort compiler supports additional (expensive) optimisations
			during the link stage. For production runs on supercomputers these
			should be used. The -TARDIS config file is the normal one to use,
			with the -HYDRA for the differing environment available on HYDRA.
	\end{description}
	To specify a default configuration file to use on a particular machine,
	create a symbolic link called \code{.default} in the config directory
	to the appropriate configuration file. This allows \code{mkconfig.py}
	to be usedwithout specifying the configuration name.

	To compile NECI, a number of linear algebra routines are required. The
	relevant routines are available in the AMD Core Math Library (ACML), the
	Intel Math Kernel Library (MKL) and in the more widely available
	combination of Basic Linear Algebra Subprograms (BLAS) and the Linear
	Algebra Package (LAPACK). The configuration files above make some
	assumptions about which packages are available, which are not always
	correct.

	In particular, the elements of the linker lines
	\begin{lstlisting}[gobble=4]
		-lacml
		-lmkl_intel_ilp64 -lmkl_core -lmkl_sequential
		-lblas -llapack
	\end{lstlisting}
	are in principle interchangeable. On personal development machines it is
	easiest to install BLAS and LAPACK, but these are generally less
	performant so are not used by default. These lines can be substituted in
	the generated \code{Makefile} before compilation.

\subsection{Building the code (Makefile system)}
	The code is built using the command
	\begin{lstlisting}[gobble=4]
		make [-j [n]] [neci.x|kneci.x|dneci.x|mneci.x|both|all]
	\end{lstlisting}
	The optional flag \code{-j} specifies that the build should be
	performed in parallel (with up to an optional number, \code{n},
	threads).

	The final argument specifies whether the normal code (\code{neci.x})
	should be built, the complex code (\code{kneci.x}), the double run
	code (\code{dneci.x}), the multiple run code (\code{mneci.x}) or
	both the normal and complex codes (\code{both}) or all of the above
	and various extra utilities
	(\code{all}).

\subsection{Git overview}
	It is essential if you plan to do developmental work to get familiar with the source-code management software `git'. The code will get unusable exponentially quickly if all development and new ideas are hacked into the master branch of the code. The nature of research is that most things probably won't work, but you want to implement them and test relatively quickly, without requiring a standard of code that will remain usable in perpetuity. To avoid an inexorable increase in code `clutter', it is essential to work in `branches' off the main code. For a more detailed introduction to the git package, see \url{git-scm.com/book/en/v2/getting-started-git-basics}. In short, the workflow should be:
	\begin{enumerate}
	\item Branch off a clean master version to implement something
	\item Test and develop in the branch
	\item Regularly merge the new code from the master branch into your personal development branch
	\item Once satisfied with the development, and that it is an improvement in scope or efficiency of the existing code, ensure it is tidy, commented, documented, as bug-free as possible, and tests added to the test suite for it. This may involve reimplementing it from a clean version of master if it can be done more efficiently
	\item Merge code back into master branch
	\end{enumerate}

	A few potentially useful git commands in roughly the workflow described above:
	\begin{description}
	\codeitem{git branch}
	See what branch I am on. -a flag for all (inc. remote) branches.
	\codeitem{git pull origin master}
	Update the master branch into the current local repository
	\codeitem{git checkout -b newbranchname}
	Fork off current branch to a new branch called `newbranchname'
	\codeitem{git commit -a -m `Commit message'}
	Commit a set of changes for the current branch to your local repository.
	\codeitem{git push origin branchname}
		Push your current local branch called branchname to a new remote
		branch of the same name to allow access to others and secure storage
		of the work
	\codeitem{git checkout -b newbranchname --track origin/remotebranch}
		Check out a branch stored on the remote repository, and allow pushing
		and pulling from the remote repository for that branch.
	\codeitem{git push}
		Push the current branch to the remote branch that it is tracking.
	\codeitem{git merge master}
		Merge the recent changes in master into your local branch (requires a
		pull first)
	\codeitem{git checkout master}
	Switch branches to the master branch
	\codeitem{git merge newbranch}
	Merge your code in `newbranch' into your current branch (potentially master)
	\end{description}

    Each commit should contain one logical idea and the commit message should
    clearly describe \emph{everything} that is done in that commit. It is fine
    for one commit to only contain a very minor change. Try and commit regularly
    and avoid large commits. It is also a good idea to make sure that code
    compiles before commiting. This helps catch errors that you may be
    introducing and also allows the use of debugging tools such as git bisect.

	It should be noted that the `stable' branch of the code, automatically
	merged into from master upon successful completion of nightly tests, is
	hosted on github on a public repository, and also pushed to the molpro
	source code. The molpro developers will quickly send us angry emails if
	poor code gets pushed into it from NECI, and I will be sure to forward
	complaints onto the relevant parties!

\section{Calculation inputs}

The NECI executable takes one input argument, which is the name of an input
file containing the instructions for carrying out the calculation. The input
file is organized in blocks, with each block being started and terminated by a
dedicated keyword. Each block can contain a number of keywords to specify
options. Here, a list of the blocks and their respective keywords is given.

The first line of the input is always \texttt{title}, the last line is always
\texttt{end}.

Some keywords are mandatory, those are marked in \textcolor{mred}{red} and are
given at the beginning of the description of each paragraph. Then come
recommended options, marked in \textcolor{oblue}{blue}, followed by further
options given in black.

\subsection{SYSTEM Block}
The SYSTEM block specifies the properties of the physical system that is
considered. The block starts with the \texttt{system} keyword and ends with
the \texttt{endsys} keyword.

\begin{description}
  \codeitem{\textcolor{mred}{system}}
  Starts the SYSTEM block. Has one mandatory additional argument to specify
  the type of the system. The options are
  \begin{description}
    \codeitem{read}
    Read in the integrals from a FCIDUMP file, used for ab-initio calculations.
    \codeitem{hubbard}
    Uses the Hubbard model Hamiltonian.
    \begin{description}
	\codeitem{k-space} In the momentum-space basis (beneficial for low $U/t$)
	\codeitem{real-space} In the real-space basis (beneficial for large $U/t$)
    \end{description}
    \codeitem{ueg}
    Uses the Hamiltonian of the uniform electron gas in a box.
  \end{description}
  \codeitem{\textcolor{mred}{endsys}}
  Terminates the SYSTEM block.
  \codeitem{\textcolor{mred}{electrons $n$, nel $n$}}
  Sets the number of electrons to $n$
  \codeitem{spin-restrict $m$}
  Sets the total $S_z$ quantum number to $\frac{m}{2}$. The argument $m$ is
  optional and defaults to 0.
  \codeitem{hphf $s$}
  Uses a basis of (anti-)symmetric combinations of Slater determinants with
  respect to global spin-flip. $s=0$ indicates anti-symmetric combinations,
  $s=1$ symmetric combinations. This is useful to exclude unwanted spin
  configurations. For example, no triplet states can occur for \texttt{hphf 0}.
  \codeitem{guga $S$}
  Activates the spin-adapted implementation of FCIQMC using CSFs (more precisely Gel'fand-Tsetlin states) and the graphical Unitary Group Approach (GUGA). The total spin $S$ must be specified in multiples of $\hbar/2$.
  So $S = 0$ is a singlet, $S = 1$ a doublet $S = 2$ a Triplet and so on.
  This keyword MUST be combined with certain \texttt{nonuniformrandexcits} input as described below! Also it is not allowed to use this keyword with the
  \texttt{spin-restrict} or \texttt{HPHF} keyword. And the number of electrons must allow the chosen spin (even number for Singlet,Triplet,etc. and odd for Doublet,Quartet,...). Additionally the choice of a reference determinant via the \texttt{definedet} keyword of the \texttt{CALC} input block (described below) is necessary in most cases, as the automatic reference state setup is not always working in a CSF-based implementation.
  \codeitem{sym $k_x$ $k_y$ $k_z$ $s$}
  Specifies the symmetry of the target state. The first three arguments set
  the momentum $(k_x,k_y,k_z)$ and are only used for Hubbard and ueg-type
  systems, the last argument $s$ specifies the irrep within $d_{2h}$ and is
  only used for ab-initio systems.
  \codeitem{lztot}
  Set the total $L_s$ quantum number. Has one mandatory additional argument,
  which is the value of $L_s$.
  \codeitem{useBrillouinTheorem}
  Assume that single excitations have zero matrix elements with the
  reference. By default, this is determined automatically.
  \codeitem{noBrillouinTheorem}
  Always assume that single excitations have nozero matrix elements with the
  reference.
  \codeitem{umatEpsilon $\epsilon$}
  Defines a threshold value $\epsilon$ below which matrix elements of the Hamiltonian are
  rounded to 0. Defaults to $10^{-8}$.
  \codeitem{diagonaltmat}
  Assume the kinetic operator is diagonal in the given basis set.
  \codeitem{noSingExcits}
  Assume there is no coupling between single excitations in the Hamiltonian.
  \codeitem{rohf}
  Use restricted open-shell integrals.
  \codeitem{read\_rofcidump}
  Read the integrals from a ROFCIDUMP file.
  \codeitem{spinorbs}
  Uses spin orbitals instead of spatial orbtials for addressing the
  integrals. This can be used if the integrals depend on the spin of the
  orbitals.
  \codeitem{molproMimic}
  Use the same orbital ordering as molpro, mimicking the behaviour of calling
  NECI from molpro. First nelec/2 orbitals  sorted by diagonal elements of the Fock matrix are considered to be occupied, and the rest to be virtual. Each sector is then sorted separately by symmetry labels.
  \codeitem{complexOrbs\_realInts}
  The orbitals are complex, but not the integrals. This reduces the symmetry
  of the 4-index integrals. Only affects \texttt{kneci} and \texttt{kmneci} calculations.
  \codeitem{complexWalkers-realInts}
  The integrals and orbitals are real, but the wave function shall be
  complex. Only affects \texttt{kneci} and \texttt{kmneci} calculations.
  \codeitem{system-replicas $n$}
  Specifies the number of wave functions that shall be evolved in
  parallel. The argument $n$ is the number of
  wave functions (replicas). Requires \texttt{mneci} or \texttt{kmneci}.
\end{description}
\subsubsection{Excitation generation options}
\begin{description}
  \codeitem{\textcolor{oblue}{nonUniformRandExcits}}
  Use a non-uniform random excitation generator for picking the move in the
  FCIQMC spawn step. This can significantly speed up the calculation. Requires an additional
  argument, that can be chosen from the following
  \begin{description}
   \codeitem{\textcolor{oblue}{pchb}}
    Generates excitations weighted directly with the matrix elements using pre-computed
    alias tables. This excitation generator is extremely fast, while
    maintaining high acceptance rates and is generally recommended when
    memory is not an issue.
    \codeitem{nosymgen}
    Generate all possible excitations, regardless of symmetry. Might have a
    low acceptance rate.
    \codeitem{4ind-weighted}
    Generate exictations weighted by a Cauchy-Schwarz estimate of the matrix
    element. Has very good acceptance rates, but is comparably slow. Using the
    \texttt{4ind-weighted-2} or \texttt{4IND-WEIGHTED-UNBOUND} instead is recommended.
    \codeitem{4ind-weighted-2}
    Generates excitations using the same Cauchy-Schwary estimate as
    \texttt{4ind-weighted}, but uses an optimized algorithm to pick orbitals
    of different spin, being faster than the former.
    \codeitem{4ind-weighted-unbound}
    Generates excitations using the same Cauchy-Schwary estimate as
    \texttt{4-ind-weighted} and optimizations as \texttt{4ind-weighted-2}, but
    uses more accurate estimates, having higher acceptance rates. This
    excitation generator has high acceptance rates at negligible memory cost.
    \codeitem{pcpp}
    The pre-computed power-pitzer excitation generator \footnote{V. Neufeld,
      A. Thom, J. Chem. Theory Comput.2019151127-140}. Has low memory cost and
    scales only mildly with system size, and can thus be used for large systems.
    \codeitem{mol-guga-weighted}
    Excitation generator for molecular systems used in the spin-adapted GUGA Approach. The specification of this excitation generator when using GUGA in ab initio systems is necessary!
    \codeitem{ueg-guga}
    Excitation generator choice when using GUGA in the UEG or k-space/real-space Hubbard model calculations. It is mandatory to specify this keyword in this case!

  \end{description}
  \codeitem{lattice-excitgen}
  Generates uniform excitations using momentum conservation. Requires the
  \texttt{kpoints} keyword.
  \codeitem{pchb-weighted-singles}
  Use a weighted single excitation generator for the pchb excitation
  generator. By default, singles are created uniformly, the weighted
  generation is much more expensive, but can help if single matrix elements
  are large.
\end{description}


\begin{description}
  \codeitem{GAS-SPEC}
    Perform a \emph{Generalized Active Spaces} (GAS) calculation and specify
      the GAS spaces.
    It is possible to select the actual implementation with the
      \texttt{GAS-CI} keyword.
    For the GAS specification it is assumed, that the orbitals
      in each GAS space are contiguous, i.e. first all orbitals
      from the first GAS space, then all from the second GAS space...
    The specification is first the number of GAS spaces $n_\text{GAS}$
      and then $3 \times n_\text{GAS}$ the specification for each GAS space
      as cumulative number of spatial orbitals and minimum and maximum
      number of particles
      $n_i^\text{cmltve}, N_i^\text{min, cmltve}, N_i^\text{max, cmltve}$.
    It is advantageous to use the line continuation (\texttt{+++})
      for human-readable formatting as table.
    Two benzenes with single inter-space excitation would be e.g. denoted as:
    \begin{table}[htbp]
    \centering
    \begin{tabular}{rrrr}
     2 & +++ \\
     6 & 5 & 7 & +++ \\
    12 &12 &12 & \\
    \end{tabular}
    \end{table}

  \codeitem{GAS-CI}
    Specify the actual implementation for GAS.
    If it is ommitted, it will be deduced from \texttt{GAS-SPEC}.
    \begin{description}
      \codeitem{GENERAL}
        Use general GAS, which is applicable to any GAS specification,
          but a bit slower than necessary for disconnected spaces.
      \codeitem{DISCONNECTED}
        Use the disconnected GAS implementations, which assumes disconnected
          spaces and performs there a bit better than the general implementation.
    \end{description}
\end{description}

  \subsubsection{Hubbard model and UEG options}
  \begin{description}
    \codeitem{lattice $type$ $l_x$ $l_y$ $l_Z$}
    Defines the basis in terms of a lattice of type $type$ with extent $l_x
    \times l_y \times l_z$. $l_x$ and $l_y$ are mandatory, $l_z$ is
    optional. $type$ can be any of \texttt{chain}, \texttt{star},
    \texttt{square}, \texttt{rectangle}, \texttt{tilted}, \texttt{triangular},
    \texttt{hexagonal}, \texttt{kagome}, \texttt{ole}.
    \codeitem{U $U$}
    Sets the Hubbard interaction strengh to $U$. Defaults to 4.
    \codeitem{B $t$}
    Sets the Hubbard hopping strengh to $t$. Defaults to -1.
    \codeitem{twisted-bc $t_1$ $t_2$}
    Use twisted boundary conditions with a phase of $t_1 \, \frac{2\Pi}{x}$
    applied along x-direction, where $x$ is the lattice size. $t_2$ is
    optional and the additional phase along y-direction, in multiples of
    $\frac{2\Pi}{y}$.
    \codeitem{open-bc $direction$}
    Set the boundary condition in $direction$ to open, i.e. no hopping across
    the cell boundary is possible. $direction$ is optional and can be one of
    \texttt{X}, \texttt{Y} or $\texttt{XY}$, for open boundary conditions in
    x-, y- or both directions. If omitted, both directions are given open
    boundary conditions. Requires a real-space basis.
    \codeitem{ueg-offset $k_x$ $k_y$ $k_z$}
    Offset $(k_x, k_y, k_z)$ for the momentum grid used in for the uniform
    electron gas.
  \end{description}

  \subsubsection{Transcorrelation options}
  \begin{description}
    \codeitem{molecular-transcorr}
    Enable the usage of a transcorrelated ab-initio Hamiltonian. This implies the
    non-hermiticity of the Hamiltonian as well as the presence of 3-body
    interactions. Requires passing the 3-body integrals in either ASCII or
    HDF5 format in a \texttt{TCDUMP} file, or \texttt{tcdump.h5},
    respectively. Enables triple excitation generation. When using this
    option, non-uniform random excitation generator become inefficient, so
    using \texttt{nonUniformRandExcits} is discouraged.
    \codeitem{ueg-transcorr $mode$}
    Enable the usage of a transcorrelated Hamiltonian for the uniform electron
    gas. This implies the non-hermiticity of the Hamiltonian as well as the
    presence of 3-body interactions. $mode$ can be one of \texttt{3-body},
    \texttt{trcorr-excitgen} or \texttt{rand-excitgen}.
    \codeitem{transcorr $J$}
    Enable the usage of a transcorrelated Hamiltonian for the real-space
    hubbard mode. This implies the non-hermiticity of the Hamiltonian. The
    optional parameter $J$ is the transcorrelation parameter and defaults to 1.0.
    \codeitem{2-body-transcorr $J$.}
    Enable the usage of a transcorrelated Hamiltonian for the momentum space
    hubbard model. This implies the non-hermiticity of the Hamiltonian as well as the
    presence of 3-body interactions. The optional argument $J$ is the
    correlation parameter and defaults to 0.25.
    \codeitem{exclude-3-body-ex}
    Disables the generation of triple excitations, but still takes into
    account 3-body interactions for all other purposes.
  \end{description}

  \subsection{CALC Block}
  The CALC block is used to set options concerning the simulation parameters
  and modes of FCIQMC. The block starts with the \texttt{calc} keyword and
  ends with the \texttt{endcalc} keyword.

  \begin{description}
    \codeitem{\textcolor{mred}{calc}}
    Starts the CALC block
    \codeitem{\textcolor{mred}{endcalc}}
    Terminates the CALC block
    \codeitem{\textcolor{oblue}{time $t$}}
    Set the maximum time $t$ in minutes the calculation is allowed to
    run. After $t$ minutes, the calculation will end.
    \codeitem{\textcolor{oblue}{nmcyc $n$}}
    Set the maximum number of iterations the calculation is allowed to
    do. After $n$ iterations, the calculation will end.
    \codeitem{seed $s$}
    Sets the seed of the random number generator to $s$. This can be used to
    specifically probe for stochastic effects, but is generally not required.
    \codeitem{averageMcExcits $x$}
    Sets the average number of spawning attempts from each walker to $x$.
    \codeitem{rdmSamplingIters $n$}
    Set the maximum number of iterations used for sampling the RDMs to
    $n$. After $n$ iterations of sampling RDMs, the calculation will end.
    \codeitem{load-balance-blocks OFF}
    Distribute the determinants blockwise in a dynamic fashion to maintain
    equal load for all processors. This is enabled by default and has one
    optional argument OFF. If given, the load-balancing is disabled.
    \codeitem{energy}
    Additionally calculate and print the ground state energy using an exact
    diagonalization technique.
    \codeitem{averageMcExcits $n$}
    The number of spawns to attempt per walker. Defaults to $1$ and should not
    be changed without good reason.
    \codeitem{adjust-averageMcExcits}
    Dynamically update the number of spawns attempted per walker. Can be used
    if the excitation generator creates a lot of invalid excitations, but
    should be avoided else.
    \codeitem{scale-spawns}
    Store the maximum value of $\frac{H_{ij}}{p_{gen}}$ for each determinant
    and use it to estimate the number of spawns per walker to prevent
    blooms. Useful when this fraction strongly depends on the determinant.
  \end{description}

  \subsubsection{Population control options}
  \begin{description}
    \codeitem{\textcolor{mred}{totalWalkers $n$}}
    Sets the targeted number of walkers to $n$. This means, the shift will be
    varied to keep the walker number constant once it reaches $n$.
    \codeitem{\textcolor{oblue}{diagShift $S$}}
    Set the initial value of the shift to $S$. A value of $S<0$ is not
    recommended, as it will decrease the population from the beginning.
    \codeitem{\textcolor{oblue}{shiftDamp $\zeta$}}
    Set the damping factor used in the shift update scheme to
    $\zeta$. Defaults to $10$.
    \codeitem{\textcolor{oblue}{stepsSft $n$}}
    Sets the number of steps per update cycle of the shift to $n$. Defaults to
    $100$.
    \codeitem{fixed-n0 $n_0$}
    Instead of varying the shift to fix the total number of walkers, keep the
    number of walkers at the reference fixed at $n_0$. Automatically sets
    \texttt{stepsSft 1} and overwrites any \texttt{stepssft} options given.
    \codeitem{targetGrowRate $grow$ $walks$}
    When the number of walkers in the calculation exceeds $walk$, the shift is
    iteratively adjusted to maintain a fixed grow rate $grow$ until reaching
    the requested number of total walkers.
    \codeitem{jump-shift OFF}
    When entering the variable shift mode, the shift will be set to the
    current projected energy. This is enabled by default. There is an optional
    argument OFF that disables this behaviour.
    \codeitem{pops-jump-shift}
    Reset the shift when restarting a previous calculation to the current
    projected energy instead of using the shift from the previous
    calculation.
    \codeitem{trunc-nopen $n$}
    Restrict the Hilbert space of the calculation to those determinants with
    at most $n$ unpaired electrons.
    \codeitem{avGrowthRate OFF}
    Average the change in walker number used to calculate the shift. This is
    enabled by default and has one optional argument OFF, which, when given,
    turns the option off.
  \end{description}

  \subsubsection{Real walker coefficient options}
  \begin{description}
    \codeitem{\textcolor{oblue}{allRealCoeff}}
    Allow determinants to have non-integer population. There is a minimal
    population below which the population of a determinant will be rounded
    stochastically. This defaults to $1$.
    \codeitem{\textcolor{oblue}{realSpawnCutoff $x$}}
    Continuous real spawning will be performed, unless the spawn has
    weight less than x. In this case, the weight of the spawning will
    be stochastically rounded up to x or down to zero, such that the
    average weight of the spawning does not change. This is a method of
    removing very low weighted spawnings from the spawned list, which
    require extra memory, processing and communication. A reasonable
    value for x is 0.01.
    \codeitem{realCoeffbyExcitLevel $n$}
    Allow all determinants up to an excitation level of $n$ to have
    non-integer population.
    \codeitem{setOccupiedThresh $x$}
    Set the value for the minimum walker weight in the main walker
    list. If, after all annihilation has been performed, any
    determinants have a total weight of less than x, then the weight
    will be stochastically rounded up to x or down to zero such that
    the average weight is unchanged. Defaults to $1$, which should only be
    changed with good reason.
    \codeitem{energy-scaled-walkers $mode$ $\alpha$ $\beta$}
    Scales the occupied threshold with the energy of a
    determinant. Has three optional arguments and requires
    \texttt{allRealCoeff}. The argument $mode$ can be one of EXPONENTIAL,
    POWER, EXP-BOUND or NEGATIVE and defaults to POWER. Both $\alpha$ and
    $\beta$ default to 1 and \texttt{realspawncutoff} $\beta$ is implied.
  \end{description}

  \subsubsection{Time-step options}
  \begin{description}
    \codeitem{\textcolor{mred}{tau $\tau$} SEARCH}
    Sets the timestep per iteration to $\tau$. Has one optional argument
    SEARCH. If given, the time-step will be iteratively updated to keep the
    calculation stable.
    \codeitem{\textcolor{oblue}{hist-tau-search $c$ $nbins$ $bound$}}
    Update the time-step based on histogramming of the ratio
    $\frac{H_{ij}}{p(i|j)}$. Not compatible with the tau $\tau$ SEARCH
    option. The three arguments $c$, $nbins$ and $bound$ are
    optional. $0<c<1$ is the fraction of the histogram used for determining
    the new timestep, $nbins$ the number of bins in the histogram and $bound$
    is the maximum value of $\frac{H_{ij}}{p(i|j)}$ to be stored.\\
    For spin-adapted GUGA calculations this option is \emph{highly} recommended! Otherwise the time-step can become quite small in these simulations.
    \codeitem{\textcolor{oblue}{max-tau $\tau_\text{max}$}}
    Sets the maximal value of the time-step to $\tau_\text{max}$. Defaults to
    $1$.
    \codeitem{min-tau $\tau_\text{min}$}
    Sets the minimal value of the time-step to $\tau_\text{min}$ and enables
    the iterative update of the time-step. Defaults to
    $10^{-7}$. The argument $\tau_\text{min}$ is optional.
    \codeitem{keepTauFixed}
    Do never update $\tau$ and the related parameter $p_\text{singles}$,
    $p_\text{doubles}$ or $p_\text{parallel}$.
    \codeitem{truncate-spawns $n$ UNOCC}
    Truncate spawns which are larger than a threshold value $n$. Both
    arguments are optional, $n$ defaults to $3$. If UNOCC is given the
    truncation is restricted to spawns onto unoccupied. Useful in combination
    with hist-tau-search.
    \codeitem{maxWalkerBloom $n$}
    The time step is scaled such that at most $n$ walkers are spawned in a
    single attempt, with the scaling being guessed from previous spawning attempts.
  \end{description}

  \subsubsection{Wave function initialization options}
  \begin{description}
    \codeitem{\textcolor{oblue}{walkContGrow}}
    When reading in a wave function from a file, do not set the shift or enter
    variable shift mode.
    \codeitem{\textcolor{oblue}{defineDet $det$}}
    Sets the reference determinant of the calculation to $det$. If no other
    initialisation is specified, this will also be the initial
    wave function. The format can either be a comma-separated list of spin
    orbitals, a range of spin orbitals (like 12-24) or a combination of both. \\
    For spin-adapted calculations using the \texttt{GUGA} keyword defining a starting reference CSF manually is \emph{highly} encouraged, as the
  	automatic way often fails. It works in similar ways as for SDs, however
  	odd numbered singly occupied orbitals indicate a positive spin coupled orbital in GUGA CSFs and even numbered singly occupied orbitals negatively spin coupled. So one needs to be careful to not define a unphysical CSF with an negative intermediate total spin $S_i < 0$. E.g. a CSF like: \\
  	\texttt{definedet  1 2 4 5}\\
  	would cause the calculation to crash, as the first singly occupied orbital (4) would cause the total spin to be negative $S_i < 0$. When defining the starting CSF one also needs to ensure that the defined CSFs satisfies the
  	total number of electrons and total $S$ defined in the \texttt{System} block of the input with the keywords \texttt{nel} and \texttt{guga}.
  	As an example a triplet (\texttt{guga 2}) CSF with 4 electrons (\texttt{nel 4}) would be \\
  	\texttt{definedet 1 2 3 5}\\
  	with the first spatial orbital doubly occupied and 2 open-shell orbitals (positively spin coupled, hence odd numbers).

    \codeitem{readPops}
    Read in the wave function from a file and use the read-in wave function
    for initialisation. In addition to the wave function, also the time-step
    and the shift are read in from the file. This starts the calculation in
    variable shift mode, maintaining a constant walker number, unless
    \texttt{walkContGrow} is given.
    \codeitem{readpops-changeref}
    Allow the reference determinant to be updated after reading in a
    wave function.
    \codeitem{startSinglePart $n$}
    Initialise the wave function with $n$ walkers on the reference only unless
    specified differently. The argument $n$ is optional and defaults to $1$.
    \codeitem{proje-changeRef $frac$ $min$}
    Allow the reference to change if a determinant obtains $frac$ times the
    population of the current reference and the latter has a population of at
    least $min$. Both arguments are optional and default to $1.2$ and $50$,
    respectively. This is enabled by default.
    \codeitem{no-changeref}
    Never change the reference.
  \end{description}

  \subsubsection{Initiator options}
  \begin{description}
    \codeitem{\textcolor{oblue}{truncInitiator}}
    Use the initiator method \footnote{D. Cleland, G.H. Booth, A. Alavi,
      J. Chem. Phys. 132, 041103 (2010)}.
    \codeitem{\textcolor{oblue}{addToInitiator $x$}}
    Sets the initiator threshold to $x$, so any determinant with more than $x$
    walkers will be an initiator.
    \codeitem{senior-initiators $age$}
    Makes any determinant that has a half-time of at least $age$ iterations an
    initiator. $age$ is optional and defaults to $1$.
    \codeitem{superInitiator $n$}
    Create a list of $n$ superinitiators, from which all connected
    determinants are set to be initiators. The superinitiators are chosen
    according to population. $n$ is optional and defaults to $1$.
    \codeitem{coherent-superInitiators $mode$}
    Apply a restriction on the sign-coherence between a determinant and any
    connected superinitiator to determine whether it becomes an initiator due
    to connection. The optional argument $mode$ can be chosen from STRICT,
    WEAK, XI, AV and OFF. The default is WEAK and is enabled by default if the
    \texttt{superInitiators} keyword is given.
    \codeitem{dynamic-superInitiators $n$}
    Updates the list of superinitiators every $n$ steps. This is enabled by default with
    $n=100$ if the \texttt{superInitiators} keyword is given. A value of 0
    indicates no update. This implies the \texttt{dynamic-core $n$} option
    (with the same $n$) unless specified otherwise.
    \codeitem{allow-signed-spawns $mode$}
    Never abort spawns with a given sign, regardless of initiators. $mode$ can
    be either POS or NEG, indicating the sign to keep.
    \codeitem{initiator-space}
    Define all determinants within a given initiator space as initiators. The
    space is specified through one of the following keywords
    \begin{description}
      \codeitem{doubles-initiator}
    Use the reference determinant and all single and double excitations
    from it to form the initiator space.
    \codeitem{cas-initiator cas1 cas2}
    Use a CAS space to form the initiator space. The parameter cas1 specifies
    the number of electrons in the cas space and cas2 specifies the
    number of virtual spin orbitals (the cas2 highest energy orbitals
    will be virtuals).
    \codeitem{ras-initiator ras1 ras2 ras3 ras4 ras5}
    Use a RAS space to form the initiator space. Suppoose the list of spatial
    orbitals are split into three sets, RAS1, RAS2 and RAS 3, ordered
    by their energy. ras1, ras2 and ras3 then specify the number of
    spatial orbitals in RAS1, RAS2 and RAS3. ras4 specifies the minimum
    number of electrons in RAS1 orbitals. ras5 specifies the maximum
    number of electrons in RAS3 orbitals. These together define the RAS
    space used to form the initiator space.
    \codeitem{optimised-initiator}
    Use the iterative approach of Petruzielo \emph{et al.} (see PRL,
    109, 230201). One also needs to use either optimised-initiator-cutoff-amp
    or optimised-initiator-cutoff-num with this option.
    \codeitem{optimised-initiator-cutoff-amp $x1$, $x2$, $x3$...}
    Perform the optimised initiator option, and in iteration $i$, choose
    which determinants to keep by choosing all determinants with an
    amplitude greater than $xi$ in the ground state of the space (see
    PRL 109, 230201). The number of iterations is determined by the
    number of parameters provided.
    \codeitem{optimised-initiator-cutoff-num $n1$, $n2$, $n3$...}
    Perform the optimised initiator option, and in iteration $i$, choose
    which determinants to keep by choosing the ni most significant
    determinants in the ground state of the space (see PRL 109, 230201).
    The number of iterations is determined by the number of parameters
    provided.
    \codeitem{fci-initiator}
    Use all determinants to form the initiator space. A fully deterministic
    projection is therefore performed with this option.
    \codeitem{pops-initiator $n$}
    When starting from a POPSFILE, this option will use the $n$ most
    populated determinants from the popsfile to form the initiator space.
    \codeitem{read-initiator}
    Use the determinants in the INITIATORSPACE file to form the initiator space.
    A INITIATORSPACE file can be created by using the write-initiator option in
    the LOGGING block.
    \end{description}
  \end{description}

  \subsubsection{Adaptive shift options}
  \begin{description}
    \codeitem{auto-adpative-shift $t$ $\alpha$ $c$}
    Scale the shift per determinant based on the acceptance rate on a
    determinant. Has three optional arguments. The first is the threshold value $t$ which
    is the minimal number of spawning attempts from a determinant over the full calculation
    required before the shift is scaled, with a default of $10$. The second is
    the scaling exponent $\alpha$ with a default of $1$ and the last is the
    minimal scaling factor, which uses $\frac{1}{\text{HF conn.}}$ as default.
    \codeitem{linear-adaptive-shift $\sigma$ $f_1$ $f_2$}
    Scale the shift per determinant linearly with the population of a determinant. All arguments are optional
    and define the function used for scaling. $\sigma$ gives the minimal
    walker number required to have a shift and defaults to $1$, $f_1$ the
    shift fraction to be applied at $\sigma$ with a default of $0$ and $f_2$
    is the shift fraction to be applied at the initiator threshold, defaults
    to $1$. Every initiator is applied the full shift.
    \codeitem{exp-adaptive-shift $\alpha$}
    Scales the shift expoentially with the population of a determinant. The
    optional argument $\alpha$ is the exponent of scaling, the default is $2$.
    \codeitem{core-adaptive-shift}
    By default, determinants in the corespace are always applied the full
    shift. Using this option also scales the shift in the corespace.
    \codeitem{aas-matele2}
    Uses the matrix elements for determining the scaling factor in the \texttt{auto-adaptive-shift}. The recommended option to scale the shift.
  \end{description}

  \subsubsection{Multi-replica options}
  \begin{description}
    \codeitem{multiple-initial-refs}
    Define a reference determinant for each replica. The following $n$ lines
    give the reference determinants as comma-separated lists of orbitals,
    where $n$ is the number of replicas.
    \codeitem{orthogonalise-replicas}
    Orthogonalise the replicas after each iteration using Gram Schmidt orthogonalisation. This will converge each
    replica to another state in a set of orthogonal eigenstates. Can be used
    for excited state search.
    \codeitem{orthogonalise-replicas-symmetric}
    Use the symmetric L{\"o}wdin orthonaliser instead of Gram Schmidt for
    orthogonalising the replicas.
    \codeitem{replica-single-det-start}
    Starts each replica from a different excited determinant.
  \end{description}

  \subsubsection{Semi-stochastic options}
  \begin{description}
    \codeitem{\textcolor{oblue}{semi-stochastic}}
    Turn on the semi-stochastic adaptation.
    \codeitem{\textcolor{oblue}{pops-core $n$}}
    When starting from a POPSFILE, this option will use the $n$ most
    populated determinants from the popsfile to form the core space.
    \codeitem{doubles-core}
    Use the reference determinant and all single and double excitations
    from it to form the core space.
    \codeitem{cas-core cas1 cas2}
    Use a CAS space to form the core space. The parameter cas1 specifies
    the number of electrons in the cas space and cas2 specifies the
    number of virtual spin orbitals (the cas2 highest energy orbitals
    will be virtuals).
    \codeitem{ras-core ras1 ras2 ras3 ras4 ras5}
    Use a RAS space to form the core space. Suppoose the list of spatial
    orbitals are split into three sets, RAS1, RAS2 and RAS 3, ordered
    by their energy. ras1, ras2 and ras3 then specify the number of
    spatial orbitals in RAS1, RAS2 and RAS3. ras4 specifies the minimum
    number of electrons in RAS1 orbitals. ras5 specifies the maximum
    number of electrons in RAS3 orbitals. These together define the RAS
    space used to form the core space.
    \codeitem{optimised-core}
    Use the iterative approach of Petruzielo \emph{et al.} (see PRL,
    109, 230201). One also needs to use either optimised-core-cutoff-amp
    or optimised-core-cutoff-num with this option.
    \codeitem{optimised-core-cutoff-amp $x1$, $x2$, $x3$...}
    Perform the optimised core option, and in iteration $i$, choose
    which determinants to keep by choosing all determinants with an
    amplitude greater than $xi$ in the ground state of the space (see
    PRL 109, 230201). The number of iterations is determined by the
    number of parameters provided.
    \codeitem{optimised-core-cutoff-num $n1$, $n2$, $n3$...}
    Perform the optimised core option, and in iteration $i$, choose
    which determinants to keep by choosing the ni most significant
    determinants in the ground state of the space (see PRL 109, 230201).
    The number of iterations is determined by the number of parameters
    provided.
    \codeitem{fci-core}
    Use all determinants to form the core space. A fully deterministic
    projection is therefore performed with this option.
    \codeitem{read-core}
    Use the determinants in the CORESPACE file to form the core space.
    A CORESPACE file can be created by using the write-core option in
    the LOGGING block.
    \codeitem{dynamic-core $n$}
    Update the core space every $n$ iterations, where $n$ is optional and
    defaults to $400$. This is enabled by default if the
    \texttt{superinitiators} option is given.
    \end{description}

  \subsubsection{Trial wave function options}
  \begin{description}
    \codeitem{\textcolor{oblue}{trial-wavefunction $n$}}
    Use a trial wave function to obtain an estimate for the energy, as
    described in \ref{sec:trial}. The argument $n$ is optional, when given,
    the trial wave function will be initialised $n$ iterations after the
    variable shift mode started, else, at the start of the calculation. The
    trial wave function is defined through one of the following keywords
    \begin{description}
    \codeitem{\textcolor{oblue}{pops-trial $n$}}
    When starting from a POPSFILE, this option will use the $n$ most
    populated determinants from the popsfile to form the trial space.
    \codeitem{doubles-trial}
    Use the reference determinant and all single and double excitations
    from it to form the trial space.
    \codeitem{cas-trial cas1 cas2}
    Use a CAS space to form the trial space. The parameter cas1 specifies
    the number of electrons in the cas space and cas2 specifies the
    number of virtual spin orbitals (the cas2 highest energy orbitals
    will be virtuals).
    \codeitem{ras-trial ras1 ras2 ras3 ras4 ras5}
    Use a RAS space to form the trial space. Suppoose the list of spatial
    orbitals are split into three sets, RAS1, RAS2 and RAS 3, ordered
    by their energy. ras1, ras2 and ras3 then specify the number of
    spatial orbitals in RAS1, RAS2 and RAS3. ras4 specifies the minimum
    number of electrons in RAS1 orbitals. ras5 specifies the maximum
    number of electrons in RAS3 orbitals. These together define the RAS
    space used to form the trial space.
    \codeitem{optimised-trial}
    Use the iterative approach of Petruzielo \emph{et al.} (see PRL,
    109, 230201). One also needs to use either optimised-trial-cutoff-amp
    or optimised-trial-cutoff-num with this option.
    \codeitem{optimised-trial-cutoff-amp $x1$, $x2$, $x3$...}
    Perform the optimised trial option, and in iteration $i$, choose
    which determinants to keep by choosing all determinants with an
    amplitude greater than $xi$ in the ground state of the space (see
    PRL 109, 230201). The number of iterations is determined by the
    number of parameters provided.
    \codeitem{optimised-trial-cutoff-num $n1$, $n2$, $n3$...}
    Perform the optimised trial option, and in iteration $i$, choose
    which determinants to keep by choosing the ni most significant
    determinants in the ground state of the space (see PRL 109, 230201).
    The number of iterations is determined by the number of parameters
    provided.
    \codeitem{fci-trial}
    Use all determinants to form the trial space. A fully deterministic
    projection is therefore performed with this option.
    \end{description}
  \end{description}

    \subsubsection{Memory options}
  \begin{description}
    \codeitem{memoryFacPart $x$}
    Sets the factor between the allocated space for the wave function and the
    required memory for the specified number of walkers to $x$. Defaults to $10$.
    \codeitem{memoryFacSpawn $x$}
    Sets the factor between the allocated space for new spawns and the estimate of
    required memory for the spawns of the specified number of walkers on a
    single processor to $x$. The memory required for spawns increases, the more
    processors are used, so when running with few walkers on relatively many
    processors, a large factor might be needed. Defaults to $3$.
    \codeitem{prone-walkers}
    Instead of terminating when running out of memory, randomly delete determinants
    with low population and few spawns.
    \codeitem{store-dets}
    Employ extra memory to store additional information on the determinants
    that had to be computed on the fly else. Trades in memory for faster iterations.
  \end{description}

  \subsubsection{Reduced density matrix (RDM) options}
  \begin{description}
    \codeitem{rdmSamplingIters $n$}
    Set the number of iterations for sampling the RDMs to $n$. After $n$
    iterations of sampling, the calculation ends.
    \codeitem{inits-rdm}
    Only take into account initiators when calculating RDMs. By default, only
    restricts to initiators for the right vector used in RDM calculation. This makes the RDMs non-variational, and the resulting energy
    is the projected energy on the initiator space.
    \codeitem{strict-inits-rdm}
    Require both sides of the inits-rdm to be initiators.
    \codeitem{no-lagrangian-rdms}
    This option disables the correction used for RDM calculation for the
    adaptive shift. Use this only for debugging purposes, as the resulting
    RDMs are flawed.
  \end{description}

  \subsubsection{METHODS Block}
  The METHODS block is a subblock of CALC, i.e. it is specified inside the
  CALC block. It sets the main algorithm to be used in the calculation. The
  subblock is started with the \texttt{methods} keyword and terminated with
  the \texttt{endmethods} keyword.
  \begin{description}
    \codeitem{\textcolor{mred}{methods}}
    Starts the METHODS block
    \codeitem{\textcolor{mred}{endmethods}}
    Terminates the METHODS block.
    \codeitem{\textcolor{mred}{method $mode$}}
    Sets the algorithm to be executed. The relevant choice for $mode$ is
    VERTEX FCIMC to run an FCIQMC calculation. Alternative choices are
    DETERM-PROJ to run a deterministic calculation and SPECTRAL-LANCZOS to
    calculate a spectrum using the lanczos algorithm.
  \end{description}

  \subsection{INTEGRAL Block}
  The INTEGRAL block can be used to freeze orbitals and set properties of the
  integrals. The block is started with the \texttt{integral} keyword and
  terminated with the \texttt{endint} keyword.
  \begin{description}
    \codeitem{integral}
    Starts the INTEGRAL block.
    \codeitem{endint}
    Terminates the INTEGRAL block.
    \codeitem{freeze $n$ $m$}
    Freeze $n$ core and $m$ virtual orbitals which are not to be
    considered active in this calculation. The orbitals are selected
    according to orbital energy, the $n$ lowest and $m$ highest orbitals in
    energy are frozen.
    \codeitem{freezeInner $n$ $m$}
    Freeze $n$ core and $m$ virtual orbitals which are not to be
    considered active in this calculation. The orbitals are selected
    according to orbital energy, the $n$ highest and $m$ lowest orbitals in
    energy are frozen.
    \codeitem{partiallyFreeze $n_\text{orb}$ $n_\text{holes}$}
    Freeze $n_\text{orb}$ core orbitals partially. This means at most
    $n_\text{holes}$ holes are now allowed in these orbitals.
    \codeitem{partiallyFreezeVirt $n_\text{orb}$ $n_\text{els}$}
    Freeze $n_\text{orb}$ virtual orbitals partially. This means at most
    $n_\text{els}$ electrons are now allowed in these orbitals.
    \codeitem{hdf5-integrals}
    Read the 3-body integrals for a transcorrelated ab-initio Hamiltonian from an HDF5
    file. Ignored when the \texttt{molecular-transcorrelated} keyword is not
    given. Requires compiling with HDF5.
    \codeitem{sparse-lmat}
    Store the 3-body integrals in a sparse format to save
    memory. Initialisation and iterations might be slower. Requires \texttt{hdf5-integrals}.
  \end{description}

  \subsection{KP-FCIQMC Block}
  This block enables the Krylov-projected FCIQMC (KPFCIQMC) method \footnote{N. S. Blunt, Ali
    Alavi, George H. Booth, Phys. Rev. Lett. 115, 050603} which is fully implemented
  in NECI. It requires \texttt{dneci} or \texttt{mneci} to be run. When
  specifying the KP-FCIQMC block, the METHODS block should be omitted. This
  block is started with the \texttt{kp-fciqmc} keyword and terminated with
  the \texttt{end-kp-fciqmc} keyword.
  \begin{description}
    \codeitem{kp-fciqmc}
    Starts the KP-FCIQMC block
    \codeitem{end-kp-fciqmc}
    Terminates the KP-FCIQMC block.
    \codeitem{num-krylov-vecs $N$}
    $N$ specifies the total number of Krylov vectors to sample.
    \codeitem{num-iters-between-vecs $N$}
    $N$ specifies the (constant) number of iterations between each Krylov
    vector sampled. The first Krylov vector is always the starting wave
    function.
    \codeitem{num-iters-between-vecs-vary $i_{12}$, $i_{23}$, $i_{34}$...}
    $i_{n,n+1}$ specifies the number of iterations between the nth and
    (n+1)th Krylov vectors. The number of parameters input should be
    the number of Krylov vectors asked for minus one. The first Krylov
    vector is always the starting wave function.
    \codeitem{num-repeats-per-init-config $N$}
    $N$ specifies the number repeats to perform for each initial
    configuration, i.e. the number of repeats of the whole evolution,
    from the first sampled Krylov vector to the last. The projected
    Hamiltonian and overlap matrix estimates will be output for each
    repeat, and the averaged values of these matrices used to compute
    the final results.
    \codeitem{averagemcexcits-hamil $N$}
    When calculating the projected Hamiltonian estimate, an FCIQMC-like
    spawning is used, rather than calculating the elements exactly,
    which would be too computationally expensive. Here, $N$ specifies the
    number of spawnings to perform from each walker from each Krylov
    vector when calculating this estimate. Thus, increasing $N$ should
    improve the quality of the Hamiltonian estimate.
    \codeitem{finite-temperature}
    If this option is included then a finite-temperature calculation is
    performed. This involves starting from several different random
    configurations, whereby walkers are distributed on random
    determinants. The number of initial configurations should be
    specified with the num-init-configs option.
    \codeitem{num-init-configs $N$}
    $N$ specifies the number of initial configurations to perform the
    sampling over. An entire FCIQMC calculation will be performed, and
    an entire subspace generated, for each of these configurations.
    This option should be used with the finite-temperature option, but
    is not necessary for spectral calculations where one always starts
    from the same initial vector.
    \codeitem{memory-factor $x$}
    This option is used to specify the size of the array allocated for
    storing the Krylov vectors. The number of slots allocated to store
    unique determinants in the array holding all Krylov vectors will be
    equal to $ABx$, where here $A$ is the length of the main walker
    list, $B$ is the number of Krylov vectors, and $x$ is the value input
    with this option.
    \codeitem{num-walker-per-site-init $x$}
    For finite-temperature jobs, $x$ specifies the number of walkers to
    place on a determinant when it is chosen to be occupied.
    \codeitem{exact-hamil}
    If this option is specified then the projected Hamiltonian will
    be calculated exactly for each set of Krylov vectors sampled,
    rather than randomly sampling the elements via an FCIQMC-like
    spawning dynamic.
    \codeitem{fully-stochastic-hamil}
    If this option is specified then the projected Hamiltonian will be
    estimated without using the semi-stochastic adaptation. This will
    decrease the quality of the estimate, but may be useful for
    debugging or analysis of the method.
    \codeitem{init-correct-walker-pop}
    For finite-temperature calculations on multiple cores, the initial
    population may not be quite as requested. This is because the
    quickest (and default) method involves generating determinants
    randomly and sending them to the correct processor at the end. It
    is possible in this process that walkers will die in annihilation.
    However, if this option is specified then each processor will throw
    away spawns to other processors, thus allowing the correct total
    number of walkers to be spawned.
    \codeitem{init-config-seeds seed1, seed2...}
    If this option is used then, for finite-temperature calculations,
    at the start of each calculation over an initial configuration,
    the random number generator will be re-initialised with the
    corresponding input seed. The number of seeds provided should be
    equal to the number of initial configurations.
    \codeitem{all-sym-sectors}
    If this option is specified then the FCIQMC calculation will be
    run in all symmetry sectors simultaneously. This is an option
    relevant for finite-temperature calculations.
    \codeitem{scale-population}
    If this option is specified then the initial population will be
    scaled to the populaton specified with the `totalwalkers' option
    in the Calc block. This is relevant for spectral calculations when
    starting from a perturbed \texttt{POPSFILE} wave function, where the
    initial population is not easily controlled.
  \end{description}
    In spectral calculations, one also typically wants to consider a particular
    perturbation operator acting on the ground state wave functions. Therefore,
    you must first perform an FCIQMC calculation to evolve to the ground state
    and output a \code{POPSFILE}. You should then start the KP-FCIQMC
    calculation from that \code{POPSFILE}. To apply a perturbation operator to
    the \code{POPSFILE} wave function as it is read in, use the pops-creation
    and pops-annihilate options. These allow operators such as
    \begin{equation}
        \hat{V} = \hat{c}_i \hat{c}_j + \hat{c}_k \hat{c}_l \hat{c}_m
    \end{equation}
    to be applied to the \code{POPSFILE} wave function. The general form is
    \code{pops-annihilate n_sum}
    \code{orb1 orb2...}
    \code{...}
    where $n_{sum}$ is the number of terms in the sum for $\hat{V}$ (2 in the
    above example), and $orbi$ specify the spin orbital labels to apply. The
    number of lines of such orbitals provided should be equal to $n_{sum}$. The
    first line provides the orbital labels for the first term in the sum, the
    second line for the second term, etc...

    \subsubsection{REALTIME Block}
    The REALTIME block enables the calculation of the real-time evolution of a
    given state and is only required for real-time calculations. Real-time
    evolution strictly requires \texttt{kneci} or \texttt{kmneci} and using
    \texttt{kmneci} is strongly recommended. This block is started with the
    \texttt{realtime} keyword and terminatd with the \texttt{endrealtime}
    keyword.

    \begin{description}
      \codeitem{realtime}
      Starts the realtime block. This automatically enables \texttt{readpops},
      and reads from a file named \texttt{<popsfilename>.0}.
      \codeitem{endrealtime}
      Terminates the REALTIME block.
      \codeitem{single $i$ $a$}
      Applies a single excitation operator exciting from orbital $i$ to
      orbital $a$ to the initial state before starting the calculation.
      \codeitem{lesser $i$ $j$}
      Calculates the one-particle Green's function $<c_i^\dagger(t)
      c_j>$. Requires using one electron less than in the popsfile.
      \codeitem{greater $i$ $j$}
      Calculates the one-particle Green's function $<c_i(t)
      c_j^\dagger>$. Requires using one electron more than in the popsfile.
      \codeitem{start-hf}
      Do not read in a popsfile but start from a single determinant.
      \codeitem{rotate-time $\alpha$}
      Calculates the evolution along a trajectory $e^{i\alpha}t$ instead of
      pure real-time trajectory.
      \codeitem{dynamic-rotation $\eta$}
      Determine a time-dependent $\alpha$ on the fly for
      \texttt{rotate-time}. $\eta$ is optional and a damping parameter,
      defaults to 0.05. $\alpha$ is then chosen such that the walker number
      remains constant.
      \codeitem{rotation-threshold $N$}
      Grow the population to $N$ walkers before starting to adjust $\alpha$.
      \codeitem{stepsalpha $n$}
      The number of timesteps between two updates of the $\alpha$ parameter
      when using \texttt{dynamic-rotation}.
      \codeitem{log-trajectory}
      Output the time trajectory to a separate file. This is useful if the
      same calculation shall be reproduced.
      \codeitem{read-trajectory}
      Read the time trajectory from disk, using a file created by NECI with
      the \texttt{log-trajectory} keyword.
      \codeitem{live-trajectory}
      Read the time trajectory from disk, using a file which is currently
      being created by NECI with the \texttt{log-trajectory} keyword. Can be
      used to create additional data for the same trajectory while the
      original calculation is still running.
      \codeitem{noshift}
      Do not apply a shift during the real-time evolution. Strongly
      recommended.
      \codeitem{stabilize-walkers $S$}
      Use the shift to stabilize the walker number if it drops below 80\% of
      the peak value. $S$ Is optional and is an asymptotic value used to fix
      the shift.
      \codeitem{energy-benchmark $E$}
      Set the energy origin to $E$ by applying a global, constant shift to the
      Hamiltonian. Can be chosen arbitrarily, but a reasonable selection can
      greatly help efficiency.
      \codeitem{rt-pops}
      A second popsfile is supplied containing a time evolved state created with the
      \texttt{realtime} keyword whose time
      evolution is to be continued. In this case, the original popsfile is
      still required for calculating properties, so two popsfiles will be
      read in.
    \end{description}



    \subsection{LOGGING Block}
    The LOGGING block specifies the output of the calculation and which
    status information of the calculation shall be collected. This block is
    started with the \texttt{logging} keyword and terminated with the
    \texttt{endlog} keyword.
    \begin{description}
      \codeitem{\textcolor{oblue}{logging}}
      Starts the LOGGING block.
      \codeitem{\textcolor{oblue}{endlog}}
      Terminates the LOGGING block.
      \codeitem{\textcolor{oblue}{hdf5-popsfile}}
      Sets the format to read and write the wave function to HDF5. Requires building
      with the \texttt{ENABLE-HDF5} cmake option.
      \codeitem{popsfile $n$}
      Save the current wave function on disk at the end of the
      calculation. Can be used to initialize subsequent calculations and
      continue the run. This is enabled by default. $n$ is optional and, when
      given, specifies that every $n$ iteration, the wave function shall be
      saved. Setting $n=-1$ disables this option.
      \codeitem{popsFileTimer $n$}
      Write out a the wave function to disk every $n$ minutes, each time
      overwriting the last output.
      \codeitem{hdf5-pops-write}
      Sets the format to write the wave function to HDF5. Requires building
      with the \texttt{ENABLE-HDF5} cmake option.
      \codeitem{hdf5-pops-read}
      Sets the format to read the wave function to HDF5. Requires building
      with the \texttt{ENABLE-HDF5} cmake option.
      \codeitem{highlyPopWrite $n$}
      Print out the $n$ most populated determinants at the end of the
      calculation. Is enabled by default with $n=15$.
      \codeitem{inits-exlvl-write $n$}
      Sets the excitation level up to which the number of initiators is logged
      to $n$. Defaults to $n=8$.
      \codeitem{binarypops}
      Sets the format to write the wave function to binary.
      \codeitem{nomcoutput}
      Suppress the printing of iteration information to stdout. This data is
      still written to disk.
      \codeitem{stepsOutput $n$}
      Write the iteration data to stdout/disk every $n$ iterations. Defaults
      to the number of iterations per shift cycle. Setting $n=0$ disables
      iteration data output to stdout and uses the shift cycle for disk
      output.
      \codeitem{fval-energy-hist}
      Create a histogram of the scaling factor used for the auto-adaptive shift
      over the energy of a determinant. Only has an effect if
      \code{auto-adaptive-shift} is used.
      \codeitem{fval-pops-hist}
      Create a histogram of the scaling factor used for the auto-adaptive shift
      over the population of a determinant. Only has an effect if
      \code{auto-adaptive-shift} is used.
    \end{description}
    \subsubsection{Semi-stochastic output options}
    \begin{description}
      \codeitem{write-core}
      When performing a semi-stochastic calculation, adding this option
      to the Logging block will cause the core space determinants to be
      written to a file called CORESPACE. These can then be further read
      in and used in subsequent semi-stochastic options using the
      \texttt{read-core} option in the CALC block.
      \codeitem{write-most-pop-core-end $n$}
      At the end of a calculation, output the $n$ most populated
      determinants to a file called CORESPACE. This can further be read
      in and used as the core space in subsequent calculations using the
      \texttt{read-core} option.
    \end{description}
    \subsubsection{RDM output options}
    These options control how the RDMs are printed. For a description of how
    the RDMs are calculated and the content of the files, please see section \ref{sec:rdms}.
    \begin{description}
      \codeitem{calcRdmOnfly $i$ $step$ $start$}
      Calculate RDMs stochastically over the course of the calculation. Starts
      sampling RDMs after $start$ iterations, and outputs an average every $step$
      iterations. $i$ indicates whether only 1-RDMs (1), only 2-RDMs (2) or
      both are produced.
      \codeitem{rdmLinSpace $start$ $n$ $step$}
      A more user friendly version of \texttt{calcrdmonfly} and \texttt{rdmsamplingiters}, this samples both
      1- and 2-RDMs starting at iteration $start$, outputting an average every
      $step$ iterations $n$ times, then ending the calculation.
      \codeitem{diagFlyOneRdm}
      Diagonalise the 1-RDMs, yielding the occupation numbers of the natural
      orbitals.
      \codeitem{printOneRdm}
      Always output the 1-RDMs to a file, regardless of which RDMs are
      calculated. May compute the 1-RDMs from the 2-RDMs.
      \codeitem{writeRdmsToRead off}
      The presence of this keyword overrides the default.  If the OFF word is present, the unnormalised \code{TwoRDM_POPS_a***}
    files will definitely not be printed, otherwise they definitely will be, regardless of the state of the
    \code{popsfile/binarypops} keywords.

    \codeitem{readRdms}
    This keyword tells the calculation to read in the \code{TwoRDM_POPS_a***} files from a previous calculation.  The
    restarted calc then continues to fill these RDMs from the very first iteration regardless of the value put with
    the \code{calcRdmOnFly} keyword.  The calculation will crash if one of the \code{TwoRDM_POPS_a***} files are missing.  If
    the \code{readRdms} keyword is present, but the calc is doing a \code{StartSinglePart} run, the \code{TwoRDM_POPS_a***} files
    will be ignored.

    \codeitem{noNormRdms}
    This will prevent the final, normalised \code{TwoRDM_a***} matrices from being printed.  These files can be quite
    large, so if the calculation is definitely not going to be converged, this keyword may be useful.

    \codeitem{writeRdmsEvery $iter$}
    This will write the normalised \code{TwoRDM_a***} matrices every $iter$ iterations while the RDMs are being
    filled.  At the moment, this must be a multiple of the frequency with which the energy is calculated.  The
    files will be labelled with incrementing values - \code{TwoRDM_a***.1} is
    the first, and then next \code{TwoRDM_a***.2} etc.
    \codeitem{write-spin-free-rdm}
    Output the spin-free 2-RDMs to disk at the end of the calculation.
      \codeitem{printRoDump}
      Output the integrals of the natural orbitals to a file.

     \codeitem{print-molcas-rdms}
     	It is now possible to calculate stochastic spin-free RDMs with the GUGA implementation. This keyword is necessary if one intends to use this feature in conjunction with \texttt{Molcas} to perform a spin-free Stochastic-CASSCF. It produces the three files \texttt{DMAT, PAMAT} and \texttt{PSMAT}, which are read-in by \texttt{Molcas}.
    \end{description}

\section{Useful References Containing Technical Details}

Original FCIQMC method:
\begin{itemize}
\item Fermion Monte Carlo without fixed nodes: a game of life, death, and annihilation in Slater determinant space. \newline
GH Booth, AJ Thom, A Alavi – The Journal of chemical physics (2009) 131, 054106
\end{itemize}

Quite a bit of symmetries some stuff on the initiator method that is actually implemented:
\begin{itemize}
\item Breaking the carbon dimer: the challenges of multiple bond dissociation with full configuration interaction quantum Monte Carlo methods.
GH Booth, D Cleland, AJ Thom, A Alavi – The Journal of chemical physics (2011) 135, 084104
\end{itemize}

Linear scaling algorithm, (uniform) excitation generation and overall algorithm of FCIQMC:
\begin{itemize}
\item Linear-scaling and parallelisable algorithms for stochastic quantum chemistry
GH Booth, SD Smart, A Alavi – Molecular Physics (2014) 112, 1855
\end{itemize}

Density matrices, real walker weights and sampling bias:
\begin{itemize}
\item Unbiased Reduced Density Matrices and Electronic Properties from Full Configuration Interaction Quantum Monte Carlo.
Catherine Overy, George H. Booth, N. S. Blunt, James Shepherd, Deidre Cleland, Ali Alavi, http://arxiv.org/abs/1410.6047
\end{itemize}

KP-FCIQMC:
\begin{itemize}
\item Krylov-projected quantum Monte Carlo
N. S. Blunt, Ali Alavi, George H. Booth, Phys. Rev. Lett. 115, 050603
\end{itemize}

    \section{Trial wave functions}
    \label{sec:trial}

    By default, NECI uses a single reference determinant, $|D_0\rangle$, in the
    projected energy estimator, or potentially a linear combination of two
    determinants if the the HPHF code is being used.
    \begin{equation}
        E_0 = \frac{\braket{D_0 | \hat{H} | \Psi}}{\braket{D_0 | \Psi}}.
    \end{equation}
    This estimator can be improved by using a more accurate estimate of the
    true ground state, a trial wave function, $\ket{\Psi^T}$,
    \begin{equation}
        E_0 = \frac{\braket{\Psi^T | \hat{H} | \Psi}}{\braket{\Psi^T | \Psi}}.
    \end{equation}
    Such a trial wave function can be used in NECI using by adding the
    \code{trial-wavefunction} option to the Calc block.
    You must also specify a
    trial space. The trial wave function used will be the ground state of the
    Hamiltonian projected into this trial space.

    The trial spaces available are the same as the core spaces available for
    the semi-stochastic option. However, you must replace \code{core} with
    \code{trial}. For
    example, to use all single and double excitations of the reference
    determinant, one should use the `doubles-trial' option.

\section{Sampling excited states with FCIQMC}

    As well as sampling the ground state, NECI can be used to estimate
    excited-state properties using an orthogonalisation procedure.
    Specifically, by performing $m$ FCIQMC simulations simultaneously,
    the lowest $m$ energy states can be sampled.

    To do this, one must use the mneci compilation. Using dneci is not
    sufficient.

    To specify how many states are to be sampled, one should use the
    \code{system-replicas} option in the System block of the input file.

    Then, the \code{orthogonalise-replicas} option should be included in
    the Calc section of the input file. This will tell NECI to orthogonalise
    the FCIQMC wave functions against each other. States representing
    high-energy wave functions will be orthogonalised against those
    representing low-energy wave functions. This prevents higher-energy
    states being projected to the ground state, and instead allows
    excited states to be converged upon.

    Also, one must tell NECI how to initialise each FCIQMC wave function.
    It is a bad idea to start from single determinants, as many of these
    will be poor estimates to the desired excited states, and so
    convergence will be slow. Instead, one should start from trial estimates
    to the desired excited states. These trial states are generated by
    calculating the lowest-energy states within a subspace.
    Thus, one must simply tell NECI what subspace to use. The options
    available are the same as for semi-stochastic and trial spaces (see above).

    For example, if one wants to initialise from the lowest-energy CISD states,
    one should use the \code{doubles-init} option in the Calc block. If one
    wants to start from the lowest-energy states in a $(10,10)$ CAS space, you
    should put \code{cas-init 10 10} in the Calc block.

    Also, single-determinant energy estimators can be give poor results for
    excited states. Instead, trial wave function-based estimators should be
    used. This should be done exactly as for the ground state -- see above for
    more details on this. For example, to use CISD wave functions in the trial
    energy estimators, include both the \code{trial-wavefunction} and
    \code{doubles-trial} options in the Calc block of the input file.

    For some example excited-state calculations with NECI, see the
    test\_suite/mneci/excited\_state directory and the tests therein.

\section{Davidson RAS code}

    NECI has an option to find the ground state of a RAS space using a
    direct CI davidson approach, which does not require the Hamiltonian to be
    stored. This code is particularly efficient for FCI and CAS spaces, but is
    less efficient for CI spaces.

    To perform a davidson calculation, put
    \begin{lstlisting}[gobble=4]
    	davidson ras1 ras2 ras3 ras4 ras5
    \end{lstlisting}
    in the Methods block, inside the Calc block. The parameters ras1-ras5 define
    the RAS space that will be used. These are defined as follows. First,
    split all of the spatial orbitals into theree sets, RAS1, RAS2 and RAS3,
    so that RAS1 contains the lowest energy orbitals, and RAS3 the highest.
    Then, ras1, ras2 and ras3 define the the number of spatial orbitals in
    RAS1, RAS2 and RAS3. ras4 defines the minimum number of electrons in RAS1.
    ras4 defines the maximum number of electrons in RAS3. These 5 parameters
    define the ras space.

    This method will allocate space for up to 25 Krylov vectors. It will iterate
    until the norm of the residual vector is less than $10^{-7}$. If this is
    not achieved in 25 iterations, the calculation will simply stop and output
    whatever the current best estimate at the ground state is.

    This code should be able to perform FCI or CAS calculations for spaces up
    to around $5\times10^6$ or so, but will probably struggle for spaces much
    larger than this.

    The method has only been implemented with RHF calculations and with $M_s=0$.

\section{RDM generation}
\label{sec:rdms}
Currently the 2-RDMs can only be calculated for closed shell systems.  However, calculation and
diagonalisation of only the 1-RDM is set up for either open shell or closed shell systems.

The original theory behind the calculation of the RDMs (including details of parallelisation) can be found in the paper:
\url{http://arxiv.org/abs/1410.6047}. The most accurate RDM method (which is also unbiased) is the double-run approach, which requires
the code to be compiled with the \code{-D__DOUBLERUN} flag in \code{CPPFLAGS} in the Makefile. This propagates
two completely independent populations of walkers, and calculates an unbiased RDM by taking cross-terms
between the two populations.

The calculation of the diagonal elements is done by keeping track of the average walker populations of each
occupied determinant, and how long it has been occupied.  The diagonal element from Di is then calculated
as <Ni>(pop1) x <Ni>(pop2) x [No. of iterations in this block], and this is included every time we start a new
averaging block, which can occur when a determinant becomes unoccupied in either population, or when we require
the calculation of the RDM energy during the simulation. As such, the exact RDM accumulated is dependent on
the interval of RDM energy calculations, but in an unbiased way.

The off diagonal elements are sampled through spawning events, and use instantaneous walker populations.
Subtle details in the code are:

1. RDMs take contributions to diagonal elements and HF connections whenever the RDM energy is calculated.
   As the averaging blocks are now reset at this point too, this change is unbiased.
2. The off-diagonal contributions (both HF connections and other contributions sampled through spawning
   events) contain contributions from both cross-terms. I.e $N_i(pop1)*N_j(pop2)$ as well as
   $N_i(pop2)*N_j(pop1)$.

There is also the facility to do a single-run calculation of the RDM.  This method is BIASED, so should
not be used for high accuracy calculations.  However, it is cheaper than the double run method, both in
memory and in simulation time, so may be useful for rough-and-ready calculations.
%With no cutoff applied
%this is very similar to the method described in DMC's thesis, which gives poor RDM energies. Note that the
%main difference is that the SR method used here resets the population averages when the RDM energy is calculated
%rather than just when the determinant becomes unoccupied. Note also that this SR method is only identical to that
%tested in CMO's thesis if the RDM energy is only calculated in the final iteration. Without this condition,
%the current implementation of SR is probably more biased in the diagonal elements than the version tested in CMO thesis,
%as squared terms are added in more regularly. However, dealing with this makes the code very untidy and requires significantly
%more memory.  As the SR method is inherently biased anyway, this is not too great a concern.
Reducing the effect of the bias in the SR method can be done by applying a cutoff to the diagonal contributions, such that
contributions are only added in if the average sign of the determinant at the time of adding in the contribution exceeds
some preset parameter.
When calculating RDMs, the RDM energy will be printed at the end of the
calculation, which is one measure of the accuracy of the RDMs.  Also printed by default are the maximum error in the
    hermiticity (2-RDM(i,j;a,b) - 2-RDM(a,b;i,j)) and the sum of the absolute errors.

\subsection{Reading in / Writing out the RDMs for restarting calculations}

	Two types of 2-RDMs can be printed out.  The final normalised hermitian 2-RDMs of the form \code{TwoRDM_a***}, or the
	binary files \code{TwoRDM_POPS_a***}, which are the unnormalised RDMs, before hermiticity has been enforced.  The
	first are the \code{2-RDM(i,j;a,b)} matrices, which are printed in
    spatial orbitals with $i<j$, $a<b$ and $i,j<a,b$.  The second are the ones to read back in if a calculation
    is restarted (they are also printed in spatial orbitals with $i<j$ and $a<b$, but for both $i,j,a,b$ and $a,b,i,j$
    because they are not yet hermitian).  These are the matrices exactly as they are at that point in the calculation.
	By default the final normalised 2-RDMs will always be printed, and the \code{TwoRDM_POPS_a***} files are connected to the
    \code{popsfile/binarypops} keywords - i.e. if a wavefunction popsfile is being printed and the RDMs are being filled,
	a RDM \code{POPSFILE} will be also.
	If only the 1-RDM is being calculated, \code{OneRDM_POPS/OneRDM} files will be printed in the same way.

\section{Performing error analysis}
    Data from an FCIQMC calculation is usually correlated. As a result,
    standard error analysis for uncorrelated data cannot be used. Instead we
    perform a so-called blocking analysis (JCP 91, 461). In this, data is
    grouped into blocks of increasing size until the data in subsequent blocks
    becomes uncorrelated, to a good approximation.

    A blocking analysis can be performed in NECI in one of two ways. Firstly,
    a rough blocking analysis is performed automatically after a job is finished.
    The final result is output to standard output and further information about
    the blocking analysis at various block sizes is output to separate files,
	such as \code{Blocks_num} and \code{Blocks_denom}. This should only be
	used as a rough
    and quick estimate as there are issues with this approach. For example, the
    analysis starts as soon as the shift is turned on. This is before the
    population has stabilised, and so unusual results can occur in the analysis
    of the denominator and numerator. Also, data is not taken from the optimal
    block size.

    A better approach for a more careful analysis is to use the blocking script
    in the utils directory, called blocking.py. The key command is
	\begin{lstlisting}[gobble=4]
		./blocking.py -f start_iter -d24 -d23 -o/ FCIMCStats
    \end{lstlisting}

	This will perform a blocking analysis starting from iteration
	\code{start_iter}.
    The analysis should be started only once the energy estimate, (column 11 in
	\code{FCIMCStats}) and the numerator and denominator (columns 24 and
	25) have
    stabilised and are fluctuating about some final value. Just because the
    energy looks stable, it does not mean that the populations is not still
    growing!

	\code{-d24 -d23'} tells the script to perform the blocking on columns 25 and
	24 of the \code{FCIMCStats} file, which correspond to the numerator and
	denominator of the energy estimator, respectively. \code{-o/} tells the script
    to also provide data for the results of dividing columns 25 and 24, which
    gives the energy estimate that we want.

    Running this will produce a graph of the errors for both the numerator and
    denominator as a function of the number of blocks (and therefore of the
    block size). As the block size increases, the error estimates should
    increase, tending towards the true values. Eventually the estimates will
    plateau. This indicates that, at this block length, the data in the blocks
    are uncorrelated to a good approximation, and the error estimate calculated
    is accurate. The data from this block length should therefore be used.

    Each estimate of the error will also have an error on it. As the block
    length increases this `error on the error' will increase. One should
    therefore use the \emph{first} block length where the plateau is reached,
    so as to minimise the error on the final error estimate.

    If no plateau is seen in the plot then the simulation has not been run for
	long enough, and needs to be continued by restarting from the \code{POPSFILE}.
    It can take on the order of $10^5-10^6$ iterations to perform an accurate
    blocking analysis.

	The \code{blocking.py} script will also output the final estimates on the energy
    at the different block lengths. You should find the blocking length where
    the errors plateau and read of the final estimates (the rightmost columns)
    from here.

	More information (including example plots, similar to those that
	\code{blocking.py} produces) is available at JCP 91, 461.

\end{document}
